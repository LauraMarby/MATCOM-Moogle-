\documentclass[11pt]{beamer}
\usetheme{Warsaw}
\usepackage[utf8]{inputenc}
\usepackage[spanish]{babel}
\usepackage{amsmath}
\usepackage{amsfonts}
\usepackage{amssymb}
\usepackage{graphicx}
%\author{}
%\title{}
%\setbeamercovered{transparent} 
%\setbeamertemplate{navigation symbols}{} 
%\logo{} 
%\institute{} 
%\date{} 
%\subject{} 
\begin{document}

%\begin{frame}
%\titlepage
%\end{frame}

%\begin{frame}
%\tableofcontents
%\end{frame}

\begin{frame}{Proyecto \textbf{Moogle!}}
\begin{flushleft}
Estudiante: \textbf{Laura Martir Beltrán}\\
Grupo: \textbf{C-113}\\

\end{flushleft}
\begin{center}
\textbf{Ciencias de la Computación} en la \textbf{Universidad de La Habana}\\

\end{center}
\begin{center}
\textbf{TEMAS:}\\
•Qué es Moogle!?\\
•Clases nuevas\\
•Operadores disponibles a utilizar\\
(para más información sobre \textbf{procesamiento de datos} y \textbf{modelo vectorial}, visitar el \textbf{informe escrito})
\end{center}

\end{frame}
\begin{frame}{•Qué es \textbf{Moogle!}?}
\begin{center}
\textbf{Moogle!} Es un motor de búsqueda el cual dada cierta información ofrecida por el usuario, devuelve todos los documentos relacionados con ella gracias a una biblioteca de datos que tiene almacenados y procesados.
\end{center}
\end{frame}

\begin{frame}{•Clases nuevas}
\begin{center}
Para una mejor implementación es necesario crear una nueva \textbf{clase} que formará una parte muy importante de proyecto.

\end{center}
\begin{center}
\textbf{1}-\textbf{TXT}(documento):\\
Características:\\
\textbf{1.1}-Título del documento(\textbf{title}).\\
\textbf{1.2}-Contenido del documento(\textbf{text}).

\end{center}

\end{frame}
\begin{frame}{•Operadores disponibles a utilizar}

Estos son signos o símbolos que indican como se debe manipular ciertos elementos de la búsqueda:\\
-[ \textbf{*} ]:  Afecta el \textbf{score} de la palabra que lo precede en la búsqueda y lo aumenta
para así darle mas importancia en los archivos por encima de las demás.\\

-[ \textbf{!} ]:  La palabra a la que preceda este \textbf{símbolo} no puede aparecer en ningún
documento devuelto en la búsqueda.\\

-[ \textbf{$\wedge$} ]:  La palabra a la que preceda este \textbf{símbolo} debe estar en todos los
documentos que devuelva la búsqueda.\\

-[ \textbf{$\thicksim$} ]:  Se encuentra entre dos palabras del \textbf{query} y define que para todos los
documentos de la base de datos, mientra más cerca entre sí estén las palabras
que preceden este símbolo,más importancia se le dará a este documento en la
entrega de un resultado.



\end{frame}

\end{document}